\documentclass[10pt]{article}
\usepackage[russian]{babel}
\usepackage[utf8x]{inputenc}
\usepackage{amsmath}
\usepackage{amssymb}
\usepackage{amsthm}
\usepackage{latexsym}
\usepackage{enumerate}
\usepackage[margin=1cm]{geometry}

\author{
	Marat Turaev\\
	\texttt{marat.turaev@hotmail.com}
	\and
	Bogdan Bugaev\\
	\texttt{bogdan.bugaev@gmail.com}
}

\begin{document}

\title{First results.}
\maketitle

\begin{enumerate}[1.]
	\item $\heartsuit$ Доказать теорему Кэли : любая конечная группа изоморфна некоторой подгруппе группы биекций некоторого множества на себя .
	\item Доказать , что в евклидовом кольце для того , чтобы натуральное p было простым , необходимо и достаточно , чтобы существовали u и v , такие , что a=qb+r , где r=0 или N(r)<N(b) .
	\item Доказать , что множество всех ортогональных матриц порядка n , в каждой строке и каждом столбце которых ровно один ненулевой элемент , равный ±1 , относительно умножения образуют полугруппу
	\item Доказать , что множество Fn×n всех матриц порядка n с определителем 1 относительно умножения образуют полугруппу
	\item Доказать , что если идеал кольца содержит обратимый элемент , то он имеет бесконечно много правых обратных .
	\item Существует ли квадратичное расширение поля Zp при p≥3 ?
	\item Доказать , что множество всех ортогональных преобразований вещественного пространства размерности n образует группу , изоморфную S3 .
	\item $\heartsuit$ Доказать , что группа Sn при n≥3 не абелева .
	\item Из теоремы Лагранжа вывести , что порядок элемента ab равен произведению порядков a и b , для которых НОК {|a|,|b|}6=|ab| .
	\item Доказать , что множество всех аффинных преобразований линейного пространства относительно умножения образует полугруппу , но не кольцо
	\item Доказать , что кольцо многочленов K[x] над областью целостности K является кольцом главных идеалов тогда и только тогда , когда n простое .
	\item Доказать , что при любом изоморфизме числовых полей подполе Q отображается тождественно , следовательно , алгебра двойных чисел есть прямое произведение двух полей вещественных чисел .
	\item В симметрической группе S5 выяснить , какие из следующих множеств являются полугруппами , а какие группу : множество векторов плоскости относительно сложения ; множество векторов пространства относительно векторного произведения ? 
\end{enumerate}

\end{document}