%%%%%%%%%%%%%%%%%%%%%%%%%%%%%%%%%%%%%%%%%
% Beamer Presentation
% LaTeX Template
% Version 1.0 (10/11/12)
%
% This template has been downloaded from:
% http://www.LaTeXTemplates.com
%
% License:
% CC BY-NC-SA 3.0 (http://creativecommons.org/licenses/by-nc-sa/3.0/)
%
%%%%%%%%%%%%%%%%%%%%%%%%%%%%%%%%%%%%%%%%%

%----------------------------------------------------------------------------------------
%	PACKAGES AND THEMES
%----------------------------------------------------------------------------------------

% \documentclass{beamer}
\documentclass[10pt, pdf,utf8,russian]{beamer}
\usepackage[T2A]{fontenc}
\usepackage[english,russian]{babel}


\mode<presentation> {

% The Beamer class comes with a number of default slide themes
% which change the colors and layouts of slides. Below this is a list
% of all the themes, uncomment each in turn to see what they look like.

%\usetheme{default}
%\usetheme{AnnArbor}
%\usetheme{Antibes}
%\usetheme{Bergen}
%\usetheme{Berkeley}
%\usetheme{Berlin}
%\usetheme{Boadilla}
%\usetheme{CambridgeUS}
%\usetheme{Copenhagen}
%\usetheme{Darmstadt}
%\usetheme{Dresden}
%\usetheme{Frankfurt}
%\usetheme{Goettingen}
%\usetheme{Hannover}
%\usetheme{Ilmenau}
%\usetheme{JuanLesPins}
%\usetheme{Luebeck}
\usetheme{Madrid}
%\usetheme{Malmoe}
%\usetheme{Marburg}
%\usetheme{Montpellier}
%\usetheme{PaloAlto}
%\usetheme{Pittsburgh}
%\usetheme{Rochester}
%\usetheme{Singapore}
%\usetheme{Szeged}
%\usetheme{Warsaw}

% As well as themes, the Beamer class has a number of color themes
% for any slide theme. Uncomment each of these in turn to see how it
% changes the colors of your current slide theme.

%\usecolortheme{albatross}
%\usecolortheme{beaver}
%\usecolortheme{beetle}
%\usecolortheme{crane}
%\usecolortheme{dolphin}
%\usecolortheme{dove}
%\usecolortheme{fly}
%\usecolortheme{lily}
%\usecolortheme{orchid}
%\usecolortheme{rose}
%\usecolortheme{seagull}
%\usecolortheme{seahorse}
%\usecolortheme{whale}
%\usecolortheme{wolverine}

%\setbeamertemplate{footline} % To remove the footer line in all slides uncomment this line
%\setbeamertemplate{footline}[page number] % To replace the footer line in all slides with a simple slide count uncomment this line

%\setbeamertemplate{navigation symbols}{} % To remove the navigation symbols from the bottom of all slides uncomment this line
}

\usepackage{graphicx} % Allows including images
\usepackage{booktabs} % Allows the use of \toprule, \midrule and \bottomrule in tables

%----------------------------------------------------------------------------------------
%	TITLE PAGE
%----------------------------------------------------------------------------------------

\title[Генератор задач]{Генератор задач по алгебраическим структурам} % The short title appears at the bottom of every slide, the full title is only on the title page

\author{
	Marat Turaev
	\and
	Bogdan Bugaev
}

\institute[SPbAU] % Your institution as it will appear on the bottom of every slide, may be shorthand to save space
{
St Petersburg Academic University \\ % Your institution for the title page
}
\date{\today} % Date, can be changed to a custom date

\begin{document}

\begin{frame}
\titlepage % Print the title page as the first slide
\end{frame}

% \begin{frame}
% \frametitle{Overview} % Table of contents slide, comment this block out to remove it
% \tableofcontents % Throughout your presentation, if you choose to use \section{} and \subsection{} commands, these will automatically be printed on this slide as an overview of your presentation
% \end{frame}

%----------------------------------------------------------------------------------------
%	PRESENTATION SLIDES
%----------------------------------------------------------------------------------------

%------------------------------------------------
\section{First Section} % Sections can be created in order to organize your presentation into discrete blocks, all sections and subsections are automatically printed in the table of contents as an overview of the talk
%------------------------------------------------

\subsection{Subsection Example} % A subsection can be created just before a set of slides with a common theme to further break down your presentation into chunks

\begin{frame}
\frametitle{Задача и решение}
Была поставлена задача генерации более или менее связных текстов задач <<домашнего задания>> по алгебре.\\
Принято решение строить марковсую цепь
\begin{itemize}
	\item Выполнить обучение на сборнике задач
	\item Генерировать фразы-задачи на этой цепи
\end{itemize}
Что удалось сделать:
\begin{itemize}
	\item Строить такую цепь
	\item Веб-интерфейс для получения наборов задач
	\item Экспортировать задачи в \LaTeX и pdf
\end{itemize}
Проблемы
\begin{itemize}
	\item Отсутсвие русских задачников в формате tex
	\item Проблемы с извлечением формул из задачников
\end{itemize}
\end{frame}

%------------------------------------------------

\begin{frame}
	\frametitle{Шедевры}
	\begin{itemize}
		\item Доказать теорему Кэли: любая конечная группа изоморфна некоторой подгруппе группы биекций некоторого множества на себя.
		\item Доказать, что множество всех ортогональных матриц порядка n, в каждой строке и каждом столбце которых ровно один ненулевой элемент, равный $\pm$1, относительно умножения образуют полугруппу.
		\item Доказать, что множество Fn$\times$n всех матриц порядка n с определителем 1 относительно умножения образуют полугруппу.
		\item Доказать, что если идеал кольца содержит обратимый элемент, то он имеет бесконечно много правых обратных.
		\item Существует ли квадратичное расширение поля Zp при p$\pm$3?
		\item Доказать, что множество всех ортогональных преобразований вещественного пространства размерности n образует группу, изоморфную S3.
	\end{itemize}
\end{frame}

%------------------------------------------------

\begin{frame}
\Huge{\centerline{The End}}
\end{frame}

%----------------------------------------------------------------------------------------

\end{document} 